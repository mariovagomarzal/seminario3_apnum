\begin{frame}{Usando \textit{splines} para parametrizar curvas}
  Hasta ahora solo sabemos como obtener la curva que interpola
  puntos de \alert{una componente}.

  Si tenemos un conjunto de $n$ puntos en $\mathbb{R}^m$ a interpolar,
  procederemos así:
  \begin{enumerate}
    \item Determinamos una partición uniforme del intervalo $[0, 1]$ en
    $n - 1$ subintervalos.
    \item Para cada componente, determinamos un \textit{spline} cúbico
    $S_i$ que interpole los puntos de la componente en los nodos de la
    partición.
    \item La curva $S: [0, 1] \longrightarrow \mathbb{R}^m$ definida por
    \[
      S(t) = \left(S_1(t), S_2(t), \ldots, S_m(t)\right)
    \]
    es una curva suave que pasa por los $n$ puntos.
  \end{enumerate}
\end{frame}
