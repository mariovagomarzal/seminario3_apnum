\begin{frame}{Interpolación segmentaria con \textit{splines}}
    
    Sea $[a, b]$ un intervalo, y sea una partición del intervalo
    \[\Delta = \{a = x_0 < x_1 < \cdots < x_n = b\}\]
    
    Definiremos el espacio de \textit{splines} como
    \[M_2^3 = {S \in C^2([a, b]) : S|_[x_{i-1}, x_ i] = q_ i(x) \in \Pi_3, i = 1,\dots, n}\]


    Por lo tanto, un \textit{spline} es una función formada por $n$ trozos polinómicos 
    de grado tres ue verifican que la función resultante es continua y su primera y su segunda derivada son tambien contínuas.
    Es decir, se han de cumplir

    \[
        q_i^{(k)}(x_i) = q_{i+1}^(k)(x_i), \quad k = 0, 1, 2    
    \]
\end{frame}